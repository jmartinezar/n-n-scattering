\documentclass{article}
\usepackage{graphicx}
\usepackage[top=2.54cm, left=2.54cm, right=2.54cm, bottom=2.54cm]{geometry}
\usepackage{amssymb}
\usepackage{amsmath}
\usepackage{caption}
\usepackage{subcaption}

\title{Scattering n-p interaction}

\begin{document}

\maketitle

The nucleon-nucleon potential is assume as:

\begin{equation}
    V_{ij} = (V_{R} + \frac{1}{2}(1 + P_{ij}^{\sigma})V_{t} + \frac{1}{2}(1 - P_{ij}^{\sigma})V_{s})(\frac{1}{2}u + \frac{1}{2}(2 - u)P_{ij}^{\tau}) + \frac{1}{4}(1 + \tau_{iz})(1 + \tau_{jz})\frac{e^{2}}{r_{ij}}
\end{equation}

take $u\approx 1$

\begin{equation}
    V_{ij} = \frac{1}{2}\left[V_{R} + \frac{1}{2}(1 + P_{ij}^{\sigma})V_{t} + \frac{1}{2}(1 - P_{ij}^{\sigma})V_{s}\right](1 + P_{ij}^{r}) + \frac{1}{4}(1 + \tau_{iz})(1 + \tau_{jz})\frac{e^{2}}{r_{ij}}
\end{equation}

in which the operator come defined as $P_{ij}^{\sigma} = (-1)^{s+1}$ and $P_{ij}^{r} = (-1)^{l}$. For $n-p$ interaction we have $\tau_{iz} = 1$ and $\tau_{jz} = -1$. We take account all possible cases to $s$ and $l$. In each case, the quantum number is in {0,1} set hence there are 4 possible combination:

\textbf{Case I:} $l=0$ y $s=0$

\begin{equation}
    V_{ij} = \frac{1}{2}(V_{R} + \frac{1}{2}(1 + (-1))V_{t} + \frac{1}{2}(1 - (-1))V_{s})(1 + 1) + \frac{1}{4}(1 + 1)(1 + (-1))\frac{e^{2}}{r_{ij}}
\end{equation}

\begin{equation}
    V_{ij} = V_R + V_s
\end{equation}

\textbf{Case II:} $l=0$ y $s=1$

\begin{equation}
    V_{ij} = \frac{1}{2}(V_{R} + \frac{1}{2}(1 + 1)V_{t} + \frac{1}{2}(1 - 1)V_{s})(1 + 1) + \frac{1}{4}(1 + 1)(1 + (-1))\frac{e^{2}}{r_{ij}}
\end{equation}

\begin{equation}
    V_{ij} = V_R + V_t
\end{equation}

\textbf{Case III:} $l=1$ y $s=0$

\begin{equation}
    V_{ij} = \frac{1}{2}(V_{R} + \frac{1}{2}(1 + (-1))V_{t} + \frac{1}{2}(1 - (-1))V_{s})(1 + (-1)) + \frac{1}{4}(1 + 1)(1 + (-1))\frac{e^{2}}{r_{ij}}
\end{equation}

\begin{equation}
    V_{ij} = 0
\end{equation}

\textbf{Case IV:} $l=0$ y $s=1$

\begin{equation}
    V_{ij} = \frac{1}{2}(V_{R} + \frac{1}{2}(1 + 1)V_{t} + \frac{1}{2}(1 - 1)V_{s})(1 + (-1)) + \frac{1}{4}(1 + 1)(1 + (-1))\frac{e^{2}}{r_{ij}}
\end{equation}

\begin{equation}
    V_{ij} = 0
\end{equation}

In this way is possible take account only two firsts cases

\begin{equation}
    V(r) =
    \begin{cases}
        V_R + V_s & \text{if} \quad l=0 \quad \text{and} \quad s=0\\
        V_R + V_t & \text{if} \quad l=0 \quad \text{and} \quad s=1\\
        0 & \text{other case.}
    \end{cases}
\end{equation}

with

\[
V_R = V_{0R} \exp(-\kappa_R r^2),
\]
\[
V_t = -V_{0t} \exp(-\kappa_t r^2),
\]
\[
V_s = -V_{0s} \exp(-\kappa_s r^2),
\]
\[
\text{with}
\]
\[
V_{0R} = 200.0 \text{ MeV}, \quad \kappa_{R} = 1.487 \text{ fm}^{-2},
\]
\[
V_{0t} = 178.0 \text{ MeV}, \quad \kappa_{t} = 0.639 \text{ fm}^{-2},
\]
\[
V_{0s} = 91.85 \text{ MeV}, \quad \kappa_{s} = 0.465 \text{ fm}^{-2}.
\]

to find $f(\theta)$ put potential in the next integral

\begin{equation}
    f(\theta) = \int_0^{\infty} r\sin{(qr)}V(r)dr
\end{equation}

for all non-vanish cases

\textbf{Case I:}

\begin{equation}
    f(\theta) = \int_0^{\infty} r\sin{(qr)}(V_R(r) + V_s(r))dr
\end{equation}

\begin{equation}
    f(\theta) = \int_0^{\infty} r\sin{(qr)}\left(V_{0R} \exp(-\kappa_R r^2) - V_{0s} \exp(-\kappa_s r^2)\right)dr
\end{equation}

\textbf{Case II:}

\begin{equation}
    f(\theta) = \int_0^{\infty} r\sin{(qr)}(V_R(r) + V_t(r))dr
\end{equation}

\begin{equation}
    f(\theta) = \int_0^{\infty} r\sin{(qr)}\left(V_{0R} \exp(-\kappa_R r^2) - V_{0t} \exp(-\kappa_t r^2)\right)dr
\end{equation}

solving integral numericaly we get $f(\theta)$, with this

\begin{equation}
    \frac{d\sigma}{d\Omega} = |f(\theta)|^2
\end{equation}

the results are shown in Figure \ref{4}

\begin{figure}[!h]
    \centering
    \begin{subfigure}[b]{0.7\textwidth}
        \centering
        \includegraphics[width=\textwidth]{4-car.pdf}
        \caption{Plot in cartesian coordinates.}
        \label{fig:first}
    \end{subfigure}
    
    \vspace{1cm}
    
    \begin{subfigure}[b]{0.7\textwidth}
        \centering
        \includegraphics[width=\textwidth]{4-polar.pdf}
        \caption{Plot in polar coordinates.}
        \label{fig:second}
    \end{subfigure}
    
    \caption{Plot of numerical results for differential cross-section in cartesian and polar coordinates.}
    \label{4}
\end{figure}


\end{document}
